\begin{abstract}\noindent
In this paper, we investigate whether network-related features of airports can add predictive power to a model predicting flight prices. We analyze the US air transport sector as a network with airports as nodes, and direct flights as edges. We find that the network is scale-free with a small subset of airports functioning as hubs in the network, and we produce a geographical map of the network with airports in their actual locations in the continental US. Secondly, we find that while the network has changed substantially over the last two decades, the hub-and-spoke nature of the network has remained, and individual airport centrality has been relatively stable. Thirdly, we find that the network is vulnerable to a removal of the airports acting as hubs, but fairly tolerant to removal of random nodes. Finally, we create a linear prediction model for flight prices to investigate whether network-related airport characteristics improve the predictive power in such a model. We employ elastic net regularization, and use K-fold cross-validation to determine the optimal hyper-parameters. We find that these features can be said to improve the prediction model only marginally, if at all.
\\
\\ \noindent
%\textbf{Keywords:}  {\textbullet}  {\textbullet}  {\textbullet}  {\textbullet} 
% \\ \\
% \textbf{Keystrokes:} 71.886 \textbf{Standard pages:} 30. \textbf{Contributions:} Thor Donsby Noe: 3.1, 3.2, 4.3, 4.4, 5.1, 5.2, 6.3, 7.1
\end{abstract}
