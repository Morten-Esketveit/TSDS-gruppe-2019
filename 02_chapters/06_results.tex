\label{sec:results}
% Consequences and implications for the public policy from different expected results.
% Network: vulnerability of network and the missing link to effect on consumers.
Our network analysis is done with data from Bureau of Transport Statistics. While this dataset is fairly comprehensive, it does not cover flights by airlines who account for less than one percent of revenues in the sector. The network is therefore not entirely 'complete' insofar as there may be routes or airports only used by airlines that do not appear in the dataset.\\
The results of our analysis of the US air transport network broadly conforms to previous results about this network. Firstly, it is a scale-free network, where the majority of airports are linked to relatively few other airports while a few airports are very well connected and act as hubs. A very basic analysis and visual inspection suggests, that the network has become larger over time. Airports tend to keep their 'role' over time, although some airports became less well connected in the period 2007-2018, while others became more well connected. This may reflect changes 'on the ground'; cities and areas that experience e.g. economic growth may attract more flights and from a larger number of airports. However, the process of how the network is 'generated' and changes over time is beyond the scope of this analysis. \\ 
Finally, the network is vulnerable to removal of hubs, but tolerant to removal of random nodes. This analysis of the vulnerability of the network is in line with previous academic work. However, it seems to be fairly hard to relate changes in network characteristics (e.g. average degree) for the air transport sector to how customers will actually be affected by a removal of these airports. The economic value of such an analysis is therefore questionable. \\

We have taken a fairly basic approach to predicting prices of flights between airports. There are a number of characteristics of the air transport sector that we do not include in our model, but that may conceivably raise predictive power. 
Firstly, an airport constitutes a natural monopoly in its geographical area. Because of this, airports are typically closely regulated. In Denmark, this regulation encompasses how much airports charge airlines for landings, luggage handling, security etc. These types of costs for the airlines likely affect prices substantially, and these costs may vary at the level of the administrative unit responsible for the regulation - perhaps at the state level. This is not included as a feature in the predictive model, but may further add predictive power. On a related note, airports are viewed as important infrastructure, and may represent large employers in their geographical area. Therefore, political concerns may cause interventions in the market that may affect prices on specific routes. Furthermore, flight prices may also be affected by the types of planes servicing specific routes, and the airlines operating the routes. To incorporate such factors into the model would require a greatly enhanced dataset, but would also likely contribute predictive power. \\
Finally, a large number of alternative node or edge characteristics exist, some of which may contribute more predictive power in the model than the ones we have chosen. 
