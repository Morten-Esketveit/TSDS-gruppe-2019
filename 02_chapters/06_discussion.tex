\label{sec:results}
% Consequences and implications for the public policy from different expected results.
% Network: vulnerability of network and the missing link to effect on consumers.
Our network analysis is done with data from Bureau of Transport Statistics. While this dataset is fairly comprehensive, it does not cover flights by airlines who account for less than one percent of revenues in the sector. The network is therefore not entirely 'complete' insofar as there may be routes or airports only used by airlines that do not appear in the dataset.\\
The results of our analysis of the US air transport network broadly conforms to previous results about this network. Firstly, it is a scale-free network, where the majority of airports are linked to relatively few other airports while a few airports are very well connected and act as hubs. A very basic analysis and visual inspection suggests, that the network has become larger over time. Airports tend to keep their `role' over time, although some airports became less well-connected in the period 2007-2018, while others became more well-connected. This may reflect changes `on the ground'; cities and areas that experience e.g. economic growth may attract more flights and from a larger number of airports. However, the process of how the network is `generated' and changes over time is beyond the scope of this analysis. \\ 
Finally, the network is vulnerable to removal of hubs, but tolerant to removal of random nodes. This analysis of the vulnerability of the network is in line with previous academic work. However, it seems to be fairly hard to relate changes in network characteristics (e.g. average degree) for the air transport sector to how customers will actually be affected by a removal of these airports. The economic value of such an analysis is therefore questionable. \\

We have taken a fairly basic approach to predicting prices of flights between airports and this is reflected in the fact that we are only able to explain around 40 pct. of the variation in the prices in our test data. There are a number of characteristics of the air transport sector that we do not include in our model, but that may conceivably raise predictive power. 
Firstly, due to large fixed cost, the airport transport sector characterized by imperfect competition. Airlines are prices setters and prices therefore differ from the marginal costs. Whereas the marginal costs are closely linked to flight duration and distance, we have only very limited proxies for demand such as the total number of flight on a given route within a year. In addition, price elasticities for demand are likely to vary across routes and time.
In relation to this, we have considered prices from only one Monday. Airlines tend to exert price discrimination depending on the day of the week. Some flights may have a reduced price on weekdays whereas others may have an increased price depending on whether the flight is considered a `recreational' flight or a `business' flight. 

Moreover, airports are viewed as important infrastructure, and may represent large employers in their geographical area. Therefore, political concerns may cause interventions in the market that may affect prices on specific routes. Furthermore, flight prices may also be affected by the types of planes servicing specific routes, and the airlines operating the routes. To incorporate such factors into the model would require a greatly enhanced dataset, but would also likely increase the predictive power of our model.

As described in section \ref{sec:background}, the presence of low cost carriers has been shown to affect flight prices. Including a variable to capture this is possible within our dataset, but requires a systematic way of demarcating low cost carriers from other airlines. 
Finally, a large number of alternative node or edge characteristics exist, some of which may contribute more predictive power in the model than the ones we have chosen. 

With the above considerations in mind the relatively poor performance of our prediction model is unsurprising. Our focus has been on investigating whether network characteristics contribute predictive power. At a basic level, this also means that an attempt to build the best possible predictive model for flight prices has not been the primary focus. While this allows us to evaluate whether including network related characteristics contributes predictive power, it also implies that the conclusions we draw may not generalize. Network related characteristics may add predictive power in our model, but not in a more comprehensive model. This is clear from our analysis; when applying a model without airport indicators, network characteristics are much more important than when airport indicators are included. 
