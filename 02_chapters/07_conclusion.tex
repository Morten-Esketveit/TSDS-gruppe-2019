\label{sec:conclusion}
In this paper, we have have analyzed the air transport sector in the United States. The sector can be analyzed as a network of airports, connected by direct flights. In this network, a minority of airports are connected to a large number of other airports, and thereby function as hubs in the network. In other words, it is a scale-free network, following a hub-and-spoke structure. The network is vulnerable to removal of these key nodes, but tolerant to removal of either random nodes or the least well connected nodes. Furthermore, we find that the number of airports has risen steadily in the past two decades, but each airport broadly speaking performs the same function (hub or spoke) in the network over time. Specifically, airports that were well-connected in 1998 tends to be well-connected in 2018 as well. We produce geographical maps of the network in each of the three years considered, with nodes in their actual location in the continental US. The hub-and-spoke nature of the network is clearly visible, and hubs seem to be fairly distributed geographically, but nonetheless placed near population centres.
\medskip\\
To investigate the importance of network structure for prices we set up a linear model for predicting flight prices. Through grid-search cross-validation, we find, that the optimal hyper-parameters in our predictive model are an l1-ratio of 1, implying that a pure Lasso regularization is preferred to a linear combination of Lasso and Ridge regularization, and that the optimal $\alpha$ is around 5.9.
Given our setup we find that network characteristics at the node level only very marginally, if at all, can be said to contribute to predictive power.