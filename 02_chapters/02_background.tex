\label{sec:background}

\subsection{US airline deregulation}
\label{subsec:b_deregulation}
The classic air transport network through most of the \nth{20} century consisted of simple point-to-point connections directly linking a small number of mayor cities as there were a relatively few number of flights overall \citep{marti2015efficiency}. Delta Airlines had established their headquarter in Atlanta which became the busiest in the US as early as 1957\footnote{\href{https://web.archive.org/web/20110301150527/http://www.atlanta-airport.com/Airport/ATL/Airport_History.aspx}{atlanta-airport.com/Airport/ATL/Airport_History.aspx}}, nonetheless, flying cross-country was still a complicated affair and it was not until the US Airline Deregulation Act of 1978 when cross-state competition was opened up and drastic changes came to the structure of the aviation industry \citep{forbes2007role, daraban2012low}. Shortly after, the hub-and-spoke structure evolved both as a business model for the individual airlines and in cooperation with regional airlines \citep{forbes2007role} as a way of increasing the frequency and coverage. Furthermore, hub-and-spoke introduced economies of scale where administration and service was less important at the spokes as as most flights frequently pass through mayor hubs.
\par
In contrast, Low Cost Carriers (LCC) emerged as well, offering point-to-point flights to secondary airports, often aiming to avoid transfers and expensive hubs \citep{daraban2012low}.

\subsection{Fare determination}
\label{subsec:b_fare}
The determinants of fares (flight prices) is investigated in a number of academic papers. In \citet{vowles2006airfare}, the author states that research has focused on three areas: The role of hub and spoke networks, airfare pricing determinants, and the role of LCC. In the aforementioned paper, the author investigates fare determinants in hub-to-hub markets. He finds, that the number of passengers, distance between airports, the share of low cost carriers, the presence of multiple airports in the geographical region and the presence of Southwest Airlines in the market or competing markets all have statistically significant effects on fares.
\par
\citet{brueckner2001model} presents a theoretical model of fare and frequency determination, and compares outcomes in a hub-and-spoke network to outcomes in a fully connected network. They find, that the hub-and-spoke network yields higher flight frequency, and higher prices for passengers whose origin or ultimate destination is a hub. The explanation for the somewhat surprising latter result is that higher frequency allows airlines to extract a higher fare from passengers in spite of lower costs.
\par
\citet{abda2012impacts} examines the impact of low cost carriers on fares in the United States. They find statistically significant effects of low cost carriers (entry or substantial growth) on flight prices.  
\medskip\\
Another obvious price determinant is distance travelled through forgone time and fuel burn. In this aspect hub-and-spoke networks are far less efficient than point-to-point flights due to transferring being both time- and fuel consuming with an extra set of landing, taxiing, waiting, and take-off as compared to cruising directly. While aviation fuel is exempted from taxes for now, simulating network construction given a carbon tax result in a higher share of direct point-to-point flights relative to transfers and stop-overs \citep{o2012fuel}.

\subsection{Modelling Airport Networks}
\citet[pp. 41-42]{costa2011analyzing} provides a brief overview on the literature of networks in the context of airports including important findings and methodology. In this literature airports are defined as nodes, and flights as edges. Typically these networks are treated as directed although most flights are back and forth between two cities. Among the highlighted findings is that the degree distribution of airport networks seem to follow the power law (this is the case for studies of the whole World, India, Brazil, and the US) which in short means that the networks are characterized by a few large hubs with a much higher degree than the average airport.\footnote{Contrary to this, \citet{he2004statistics} finds that the Chinese network of airports does not follow the power law.} It is further suggested that the topology of the network is primarily determined by GDP and population size across cities in the network.
\medskip\\
\citet{chi2004structural} show the US airport network is affected by "errors" and "attacks". Their dataset covers 215 US airports, and flights in a specific week. To simulate an attack, they successively remove airports in order of importance, starting with the most connected airport. Conversely, to simulate an error, the authors remove airports in order of fewest connections. They then investigate how errors and attacks respectively influence key topological measures, specifically the average degree, the clustering coefficient, the diameter and efficiency. They find, that these measures are affected far more by removal of the most well connected airports, compared to removal of the least connected airports.
\medskip\\
\citet{rocha2017dynamics} outlines the fundamental properties of airport networks and surveys the literature on the dynamic modelling of these networks. He divides the research on dynamic modelling into two categories one in which long-term structural changes are analyzed using network ``snapshots'' and one that focuses on short-term changes which is used to analyze how delays propagate through the network and the logistics of the airport industry. Considering the first category, the literature mainly focuses on changes in basic network statistics over time. First, it is described how the deregulation of the American transportation sector changed the airport network from point-to-point to hub-and-spoke systems. Even though the airport networks have changed continually since then, the degree distribution seems to be quite constant. On the other hand, betweenness centrality and the clustering structure of the network seems to change over time.
\par
Looking at the short-term dynamics, it is found that airports that may be close in the static network can be quite far away from each other in a dynamic network since some routes are operated very infrequently.  


