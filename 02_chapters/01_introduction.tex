\label{sec:intro}
% Idea: How would a geographically confined natural disaster affect the network (e.g. if it was in Atlanta?). 
% Transmission of delays?

% Motivation of the study: why you focus on this particular issue
% Hypothesis and objective(s)
% Description of the background (theoretical and empirical) that lead you to propose the hypothesis
% Approach and summary of results: what is your strategy to check the hypothesis and the main result
% Structure of the paper

%Firstly we characterize the air transport sector as a network, with airports as nodes and flights as edges. Secondly, we combine results from this network-analysis with spatial and other data and attempt to predict prices on flights.

% NOTER:
% Lidt mere motiverende i indledningen.
% Afsnittet economic theory burde nærmere hedde Network Theory.

Air transport facilitates international trade, foreign direct investment and tourism between and within countries. According to the US Federal Aviation Administration (\citet{FAA}), aviation and the air transport sector in 2014 contributed 1,6 trillion dollars in total economic activity to the US economy and supported nearly 11 million jobs. Meanwhile, the network of airports connected through direct flights is a textbook example of a scale-free network characterized by a hub-and-spoke structure. In this paper, we firstly describe the basic characteristics of this network, investigate the hub-and-spoke nature of the network including how the network is distributed geographically, assess the vulnerability of the network, and describe how the basic features of the network have changed over time. Secondly, we investigate whether network-related characteristics at the airport level contribute predictive power in a model to predict flight prices. 

\medskip 

We begin with a basic characterization of the network. We find support for a hub-and-spoke view of the network, with a minority of airports being exceedingly well connected in the network and the majority of airports having direct flights to a relatively small number of airports. In other words, the network is scale-free, which is also apparent from the degree distribution of the network. We then take advantage of the geographical nature of the network, and produce a map of the network with nodes at their actual locations in the United States. \\ Furthermore, we conduct an analysis of how network characteristics are affected by a removal of certain nodes from the network, in order to assess the vulnerability of the network. In line with the results found in (\cite{chi2004structural}), we find that the network is vulnerable to removal of hubs (well connected airports), but reasonably tolerant to removal of random nodes and removal of the least connected nodes. \\ Finally, we investigate how the network has changed over time. 

\medskip

The network analysis yields a number of airport-level characteristics that expand the feature set used in the second part of the paper. In the second part of the paper, we attempt to predict flight prices using machine learning methods. The key question we wish to answer is, whether the network-related features add predictive power to the model. A priori, one might reasonably expect that the network characteristics matter for the airlines' planning of flights, and are informative of the competitive environment on a given route, both of which may affect flight prices. We find, that network characteristics only to an extremely limited extent improves the predictive power. 

\medskip
The remainder of the paper is structured as follows. In section \ref{sec:background} we briefly review related academic literature. Next, we provide the theoretical background in section \ref{sec:theory}, with an emphasis on the network theory that we draw upon. We then describe the data used in the analysis in section \ref{sec:data}, before presenting the results of the analysis in section \ref{sec:empirical}. In section \ref{sec:results} we discuss the results of the analysis, and we conclude the paper in section \ref{sec:conclusion}.
