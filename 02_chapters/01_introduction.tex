\label{sec:intro}
% Idea: How would a geographically confined natural disaster affect the network (e.g. if it was in Atlanta?). 
% Transmission of delays?

% Motivation of the study: why you focus on this particular issue
% Hypothesis and objective(s)
% Description of the background (theoretical and empirical) that lead you to propose the hypothesis
% Approach and summary of results: what is your strategy to check the hypothesis and the main result
% Structure of the paper

%Firstly we characterize the air transport sector as a network, with airports as nodes and flights as edges. Secondly, we combine results from this network-analysis with spatial and other data and attempt to predict prices on flights.
In this paper we investigate the air transport sector in the United States.\\ We begin by characterizing the network of airports, where each airport is connected to other airports through direct flights. We find support for a hub-and-spoke view of the network, with a minority of airports being disproportionately well connected in the network and the majority of airports having direct flights to a relatively small number of airports. In other words, the network is scale-free, which becomes clear from the degree distribution of the network. \\ Furthermore, we conduct an analysis of how network characteristics are affected by a removal of certain nodes from the network, in order to assess the vulnerability of the network. In line with the results found in \cite{chi2004structural}, we find that the network is vulnerable to removal of hubs (well connected airports), but reasonably tolerant to removal of random nodes and removal of the least connected nodes. \\ Finally, we take advantage of the geographical nature of the network, and produce a map of the network with nodes at their actual locations in the United States. 

The network analysis yields a number of airport-level characteristics that expand the feature set used in the second part of the paper. In the second part of the paper, we attempt to predict flight prices using (...). The key question we wish to answer is, whether the network-related features add predictive power to the model. A priori, one might reasonably expect, that the network characteristics matter for the airlines' planning of flights, and are informative of the competitive environment on a given route, both of which may affect flight prices. 

We find that (...)
